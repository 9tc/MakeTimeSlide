\documentclass[dvipdfmx]{beamer}
%\documentclass[dvipdfm,cjk]{beamer} %% オプションは環境や利用するプログラムに
%\documentclass[dvips,cjk]{beamer}   %% よって変える

\usepackage{bxdpx-beamer}% dvipdfmxなので必要
\usepackage{pxjahyper}% 日本語で'しおり'したい
\usepackage{minijs}% min10ヤダ
\usepackage{comment}
\usepackage{color}
\renewcommand{\kanjifamilydefault}{\gtdefault}% 既定をゴシック体に
\renewcommand{\figurename}{図}
\renewcommand{\tablename}{表}

\AtBeginDvi{\special{pdf:tounicode 90ms-RKSJ-UCS2}}

\setbeamertemplate{navigation symbols}{}
\usetheme{Boadilla}
%\usecolortheme[RGB={30,30,137}]{structure}

%\usefonttheme{professionalfonts}       %% 数式の文字を通常の LaTeX と同じに

\setbeamertemplate{theorems}[numbered]  %% 定理に番号をつける
\newtheorem{thm}{Theorem}[section]
\newtheorem{proposition}[thm]{Proposition}
\theoremstyle{example}
\newtheorem{exam}[thm]{Example}
\newtheorem{remark}[thm]{Remark}
\newtheorem{question}[thm]{Question}
\newtheorem{prob}[thm]{Problem}

\definecolor{ired}{rgb}{0.72, 0.33, 0.31}
\definecolor{iblue}{rgb}{0.42, 0.56, 0.75}

\setbeamertemplate{itemize item}{$\vartriangleright$}
\setbeamertemplate{itemize subitem}{$\blacktriangleright$}
\setbeamertemplate{itemize subsubitem}{$\blacktriangleright$}

\begin{document}
\title[時間を作る思考]{時間を作る思考}
\author[dokaraya]{dokaraya}
\date{\today}

\begin{frame}
\titlepage
\nocite{*}
\end{frame}

\begin{frame}
  \frametitle{目的と結論}
  \begin{block}{目的}
    \begin{itemize}
      \item 時間を作るための具体的な方法を学ぶ
    \end{itemize}
  \end{block}

  \begin{exampleblock}{結論}
    \begin{itemize}
      \item 時間を作るための時間を作る
      \item 「選択と集中」の戦略を用いる
      \item 自分の時間を大切にする
    \end{itemize}
  \end{exampleblock}
\end{frame}

\begin{frame}[t]{時間を作るための時間を作る}
  \begin{block}{毎日決めるべきこと}
    \begin{itemize}
      \item 今日の最優先事項は何か?
      \begin{itemize}
        \item 緊急性の高い事項は無いか?
        \item 今日を満足した一日にするためにやりたいことは?
        \item 今日を振り返る時喜びを感じるためにやりたいことは?
      \end{itemize}
      \item 最優先事項の具体的なゴールは?
      \item 最優先事項の障害となるものは?その対処法は?
    \end{itemize}
  \end{block}

  \begin{exampleblock}{毎日やるべきこと}
    \begin{itemize}
      \item エネルギーの補充をする
      \begin{itemize}
        \item よい食事・よい運動・よい時間
      \end{itemize}
      \item 今日の最優先事項は達成できたか考える
      \begin{description}[] \itemsep 0pt \parskip 0pt \parsep 0pt \leftmargin 10pt
        \item[\alert{Yes}] すばらしい!自分で自分を褒めて!
        \item[\structure{No}] 何が障害だったか?その対処案は?
      \end{description}
    \end{itemize}
  \end{exampleblock}
\end{frame}

\begin{frame}{「選択と集中」の戦略を用いる}
  \begin{block}{選択と集中}
    \begin{itemize}
      \item 最優先事項を選択したらそれに集中する
      \begin{itemize}
        \item 集中を妨げるものを抑制する
        \begin{itemize}
          \item 自身からスマホを遠ざける
          \item メールや時事のチェックを後回しにする
          \item 今やっていること以外のことが気になったらメモする
          \item 雑務は「雑務をやる時間」を決めてから
        \end{itemize}
      \end{itemize}
    \end{itemize}
  \end{block}
\end{frame}

\begin{frame}{自分の時間を大切にする}
  \begin{block}{大切にすべき時間の例}
    \begin{itemize}
      \item 選択した事項を実行する時間
      \item 大切な人と触れ合う時間
      \item 習慣となった良い時間
    \end{itemize}
  \end{block}

  \begin{exampleblock}{行うべきこと}
    \begin{itemize}
      \item それらを中心にスケジュールを組む
      \item それ以外の時間をなるべく短くする
      \item 時間を浪費するものに引き寄せられない
    \end{itemize}
  \end{exampleblock}


\end{frame}

\begin{frame}
  \frametitle{目的と結論(再掲)}
  \begin{block}{目的}
    \begin{itemize}
      \item 時間を作るための具体的な方法を学ぶ
    \end{itemize}
  \end{block}

  \begin{exampleblock}{結論}
    \begin{itemize}
      \item 時間を作るための時間を作る
      \item 「選択と集中」の戦略を用いる
      \item 自分の時間を大切にする
    \end{itemize}
  \end{exampleblock}
  \begin{itemize}
    \item それではよい時間を!
  \end{itemize}
\end{frame}

\begin{frame}[allowframebreaks]{参考文献}
    \scriptsize
    \beamertemplatetextbibitems
    \bibliographystyle{junsrt}
    \bibliography{ref}
\end{frame}
\end{document}
